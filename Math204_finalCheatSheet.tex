\documentclass[8pt,landscape]{article}
\usepackage{multicol}
\usepackage{calc}
\usepackage{ifthen}
\usepackage[landscape]{geometry}
\usepackage{hyperref}
\usepackage{amsmath}

\ifthenelse{\lengthtest { \paperwidth = 11in}}
	{ \geometry{top=.5in,left=.5in,right=.5in,bottom=.5in} }
	{\ifthenelse{ \lengthtest{ \paperwidth = 297mm}}
		{\geometry{top=1cm,left=1cm,right=1cm,bottom=1cm} }
		{\geometry{top=1cm,left=1cm,right=1cm,bottom=1cm} }
	}

% Turn off header and footer
\pagestyle{empty}
 

% Redefine section commands to use less space
\makeatletter
\renewcommand{\section}{\@startsection{section}{1}{0mm}%
                                {-1ex plus -.5ex minus -.2ex}%
                                {0.5ex plus .2ex}%x
                                {\normalfont\large\bfseries}}
\renewcommand{\subsection}{\@startsection{subsection}{2}{0mm}%
                                {-1explus -.5ex minus -.2ex}%
                                {0.5ex plus .2ex}%
                                {\normalfont\normalsize\bfseries}}
\renewcommand{\subsubsection}{\@startsection{subsubsection}{3}{0mm}%
                                {-1ex plus -.5ex minus -.2ex}%
                                {1ex plus .2ex}%
                                {\normalfont\small\bfseries}}
\makeatother

% Define BibTeX command
\def\BibTeX{{\rm B\kern-.05em{\sc i\kern-.025em b}\kern-.08em
    T\kern-.1667em\lower.7ex\hbox{E}\kern-.125emX}}

% Don't print section numbers
\setcounter{secnumdepth}{0}


\setlength{\parindent}{0pt}
\setlength{\parskip}{0pt plus 0.5ex}


% -----------------------------------------------------------------------

\begin{document}

\raggedright
\footnotesize
\begin{multicols}{3}


% multicol parameters
% These lengths are set only within the two main columns
%\setlength{\columnseprule}{0.25pt}
\setlength{\premulticols}{1pt}
\setlength{\postmulticols}{1pt}
\setlength{\multicolsep}{1pt}
\setlength{\columnsep}{2pt}

\begin{center}
     \Large{\textbf{MATH 204 Cheat Sheet}} \\
\end{center}
\subsection{Simple Linear Regression}
 $$
        Y_i = \beta_0 + \beta_1 x_i + \epsilon_i
 $$
 where
 $$
\hat{\beta}_1=\frac{\sum_{i=1}^{n}(y_i-\bar{y})(x_i-\bar{x})}{\sum_{i=1}^{n}(x_i-\bar{x})^2}=\frac{S_{xy}}{S_{xx}}
 $$
 $$
        \hat{\beta}_0=\bar{y}-\hat{\beta}_1\bar{x}
 $$
 $\beta_1$ is the change in the mean of $Y_i$ for a 1 unit increase in $x_i$, $\beta_0$ is the mean when $x_i= 0 $ 
 
 $S_{XX} = \sum (x_i - \bar{x} )^2$, $S_{YY} =  \sum (y_i - \bar{y} )^2$, $S_{XY} = \sum (x_i - \bar{x} )(y_i - \bar{y} )$
 
 \subsection{Estimating $\sigma^2$}
 \begin{enumerate}
 \item Standard deviation of $\hat{\beta}_1$: $\sigma_{\hat{\beta}_1} =\sqrt{var(\hat{\beta}_1)} = \sigma / \sqrt{S_{XX}}$
 \item Variance of residuals: $\hat{\sigma}^2=\frac{1}{n-2}\displaystyle\sum_{i=1}^{n}(y_i-\hat{y}_i)^2=\frac{SSE}{n-2}$
 \item $SSE = S_{YY} - \hat{\beta}_1 S_{XY}$
 \item $\hat{\sigma}_{\hat{\beta}_1} = \hat{\sigma}/\sqrt{S_{XX}}$
 \end{enumerate}
 \subsection{Inference about $\beta_1$}
\begin{enumerate}
\item When the error terms are normal, $\hat{\beta}_1 \sim \mathcal{N}(\beta_1, \sigma^2/ S_{XX})$
\item $T = \frac{\hat{\beta}_1 - \beta_1}{\hat{\sigma}/S_{XX}} \sim t_{n-2}$

$$\mathcal{H}_0 : \beta_1 = 0 \quad vs \quad \mathcal{H}_a : \beta_1 \neq 0$$
$T_{obs} = \frac{\hat{\beta}_1}{\hat{\sigma}_{\hat{\beta}_1}} = \frac{ \hat{\beta}_1}{\hat{\sigma}/\sqrt{S_{XX}}}$

Compare $T_{obs}$ with the student distribution $t_{n-2, \alpha/2}$ to get RR.
\item Could get same conclusion from p-value, which illustrates the probability that our results occurred under $\mathcal{H}_0$. That is, the probability that $F>\mathcal{F}$ under $\mathcal{H}_0.$

\item Confidence interval for $\beta_1$: $\hat{\beta}_1 \pm t_{n-2, \alpha/2 } \frac{\hat{\sigma}}{\sqrt{S_{XX}}}$.
\end{enumerate}  
\subsection{ANOVA}
\begin{enumerate}
\item $SS_{reg} = S_{YY} - SSE = \sum (y_i - \bar{y})^2 - \sum (y_i - \hat{y}_i) ^2 = \sum(\hat{y}_i - \bar{y} )^2$
\item $T \sim t_v, \quad T^2 \sim \mathcal{F}(1, v)$, where the latter is the Fisher-Snedecor dis.
\item ANOVA table guide:
\begin{itemize}
	\item (X, Sum Sq) = $SS_{reg}$
	\item (Residuals, Sum Sq) = SSE
	\item (Residuals, Df) = $n-2$
\end{itemize}
\item lm summary table
\begin{itemize}
\item t-value (slope): $T_{obs} = \frac{ \hat{\beta}_1}{\hat{\sigma}_{\hat{\beta}_1}}$
\item F-statistic : $T^2_{obs} $
\item Residual std error: $\hat{\sigma}$
\end{itemize}
\end{enumerate}
\subsection{Correlation}
\begin{enumerate}
\item corr(X,Y) = corr(Y,X)
\item $r = S_{XY} /\sqrt{S_{XX}S_{YY}}$ is an estimator for $\rho$ (the true pop. correlation).
\item $(1-\alpha)100\%$ confidence interval for $\rho$: 
transform $r$ to $z=0.5 \ln(\frac{1+r}{1-r})$. Build an interval:$z \pm \frac{z_{\alpha/2}}{\sqrt{n-3}}=(c_l, c_u)$, where $z_{\alpha/2}$ is from the standard Normal table. Then, the interval is
        $\left( \frac{e^{2c_L}-1}{e^{2c_L}+1},\frac{e^{2c_U}-1}{e^{2c_U}+1} \right)$
\item Coefficient of determination: $R^2 = 1 - SSE/S_{YY}$
\end{enumerate}
\subsection{Estimating response}
\begin{enumerate}
\item Mean response confidence interval: $\hat{y}_0 \pm t_{n-2, \alpha/2 } \hat{\sigma} \sqrt{ 1/n + (x_0 - \bar{x} )^2/ S_{XX}}$
\item Individual value $Y_0$ confidence interval: $\hat{y}_0 \pm t_{n-2, \alpha/2} \hat{\sigma}\sqrt{ 1+1/n + (x_0 - \bar{x} )^2/ S_{XX}}$
\end{enumerate}

\subsection{Residual Analysis} 
\begin{enumerate}
\item Assumptions: $\epsilon_i$ are independent, $E(\epsilon_i) = 0, \; var(\epsilon_i) = \sigma^2, \; \epsilon_i \sim \mathcal{N} (0, \sigma^2)$
\item Check Normality with QQ plot and histogram of the studentized residuals, which have mean 0, all residuals should lie within 3 std deviations.
\item Check $E(\epsilon_i) = 0$ by plotting studentized residuals against fitted values. Points should have equal variance and zero mean, i.e. evenly distributed.
\end{enumerate}

\subsection{Polynomial Regression}
$Y_i = \beta_0 + \beta_1x_i + \beta_2x^2_i + ... + \beta_p x^p_i\epsilon_i $, not all intermediate powers need be present.

Higher-order terms are specified using the $I(\cdot) $ function in R.

1. Test that the quadratic term is zero: $H_0 :\beta_2 =0.$

2. If rejected, use  linear and quadratic terms in model.

3. If not rejected, there is no evidence that the quadratic model gives significant improvement over the linear model.

\subsection{Multiple Regression (2+ covariates)}
$Y_i = \beta_0 + \beta_1 x_{i1} + \beta_2 x_{i2} + ... + \beta_k x_{ik} + \epsilon_i$

The model is linear in the parameters $(\beta_i)$, not necessarily in the covariates $(x_i)$. Same assumptions are made about the residuals. 

$\beta_j$ is the change in the mean of $Y_i$ for a 1 unit increase of $x_{ij}$ when holding all other variables constant.

\begin{enumerate}
\item $\hat{\sigma}^2 = (n - (K+1)) ^{-1} \sum (y_i -\hat{y}_i)^2 = SSE/(n-(K+1))$ where (K+1) is the number of coefficients $\beta_i$ in the model. 
\item Can test each coefficient individually with same hypothesis as in simple regression. In which case, we test for e.g. $\beta_j$ after adjusting for all other variables.
\item Confidence interval for $\beta_j$ : $\hat{\beta}_j \pm t_{n-(K+1), \alpha/2} \cdot \hat{\sigma}_{\hat{\beta}_j}$
\item Global Fit

$R^2_a = 1 - \frac{n-1}{n-(K+1)}\left(\frac{SSE}{S_{YY}}\right) = 1 - \frac{n-1}{n-K-1}(1-R^2) $

e.g. if $R^2_a = 0.80$, then we say that the model explains 80\% of the variance in Y.

\textbf{Overall hypothesis:}
$$\mathcal{H}_0 : \beta_1 = \beta_2 = ... = 0 \quad \mathcal{H}_a : \text{ at least one } \beta_j \neq 0$$
$F_{statistic} = \frac{(S_{YY} - SSE ) /K }{SSE/(n-(K+1)} = \frac{R^2/ K}{(1-R^2)/(n-(K+1))}$

$\mathcal{H}_0$ is rejected for $F> \mathcal{F}_{\alpha, K, n-(K+1)}$.

\end{enumerate}
\subsection{Interaction}
if an interaction is suspected between $X_1$ and $X_2$, we incorporate the interaction by setting
\begin{align*}
Y_i &= \beta_0 + \beta_1x_{i1} + \beta_2x_{i2} + \beta_3 x_{i1} x_{i2} + \epsilon_i \\ 
	&= \beta_0 + \beta_1x_{i1} +(\beta_2 + \beta_3x_{i1})x_{i2} + \epsilon_i \\
	&= \beta_0 + (\beta_1 +\beta_3x_{i2})x_{i1} + \beta_2 x_{i2}+ \epsilon_i 
\end{align*}
In the above model, a 1-unit increase in $x_2$ for a fixed $x_1$ corresponds to an estimated $\hat{\beta}_2 + \hat{\beta}_3 x_1$ increase in $Y_i$.

1) Fit the model including the covariates and interaction.

2) Conduct a global F-test with $\mathcal{H}_0 : \beta_1 = \beta_2 = \beta_3 = 0$

3) If rejected, test for an interaction by using a Student t-test to test $\mathcal{H}_0 : \beta_3 = 0$. If rejected, stop. Otherwise, re-fit the model without the interaction.

\subsection{Qualitative}
Set $Z_i = 0 \; \forall i$ for reference group and $Z_i =\left\{ \begin{matrix} 1 &\text{if condition i} \\ 0 & \text{otherwise} \end{matrix} \right.$
$$Y_i = \beta_0 + \beta_1 z_1 + \beta_2z_2 + \epsilon_i$$
Where $\hat{\beta}_0 = \mu_0$, $\hat{\beta}_1 = \mu_1 -\mu_0$, and $\hat{\beta}_2 = \mu_2- \mu_0$, and $\mathcal{H}_0 : \beta_1 = \beta_2 = 0 \iff \mathcal{H}_0 :\mu_2 = \mu_1 = \mu_0$, $\mathcal{H}_a:$ at least 1 $\beta_i\neq 0$

\textbf{Qualitative and quantitative}:

The model is $Y_i = \beta_0 + \beta_1z_i + \beta_2 x_i$. 

1) $z_i = 0$: $Y_i = \beta_0 + \beta_2x_i$

2) $z_i = 1$: $Y_i = \beta_0 + \beta_1 + \beta_2x_i$

So the slope is the same, only y-intercept changes.

Interaction: $Y_i = \beta_0 + \beta_1z_i + \beta_2 x_i + \beta_3 z_1 x_i$. Then, slopes vary:

1) $z_i = 0$: $Y_i = \beta_0 + \beta_2x_i$

2) $z_i = 1$: $Y_i = \beta_0 + \beta_1 + (\beta_2 + \beta_3)x_i$

Where $\beta_1 + \beta_3 x_i$ is the difference in $Y$ between $z_i=1$ and $z_i = 0$. Should always test for existence of an interaction.
If no evidence to reject $\mathcal{H}_0$ of no interaction, must re-fit model without interaction.
If evidence of interaction, slopes are different and interpret the results accordingly.
\subsection{Comparing Nested Models}
$M_0$ and $M_1$ are nested models if one contains a subset of the other.
$M_0 = \beta_0+...+ \beta_gx_g$, $M_1 = M_0+ \beta_{g+1}x_{g+1}+...+\beta_kx_k$
$$\mathcal{H}_0: \beta_{g+1}x_{g+1} = ...=\beta_kx_k=0 \quad \mathcal{H}_a : \text{ at least 1 } \beta_i\neq 0$$
Note: we always have $SSE_{M_0}\geq SSE_{M_1} $. 

1) $SSE_{M_0}- SSE_{M_1} $ large $\implies$ $M_1$ explains more variance than just using $M_0$.

2)$SSE_{M_0}- SSE_{M_1} $ small $\implies$ additional terms in $M_1$ don't contribute to model fit.

To determine how "large" the difference is:

$$F = \frac{(SSE_{M_0}- SSE_{M_1}) /(k-g)}{SSE_{M_1} / (n - (k+1))}$$
We reject $\mathcal{H}_0$ if $F > \mathcal{F}(\alpha,k-g, n-(k+1))$ 

\subsection{Multicollinearity} 
When two covariates in a regression analysis are highly correlated with each other (their coefficient of correlation is high), the analysis is said to be subject to multicollinearity. The covariates “compete” for the explanatory power in the association with the response.

\subsection{Multinomial Distribution}
One qualitative variable $C$ can take $k$ possible values $\{c_1, ... ,c_k\}$. Let $X_i$ the \# of times $c_i$ occurs. The set of $X_i$ has a multinomial distribution.
$$P(X_1 = n_1 , ... , X_k = n_k) = \frac{n!}{n_1!... n_k!} p_1^{n_1}...p_k^{n_k}$$ 
where $n_1 + ... + n_k = n$. $E(X_i) = np_i$. 
\subsubsection{Chi-square}
$\mathcal{H}_0 : p_1 = p^*_1 ,..., p_k = p^*_k \quad \mathcal{H}_a : p_i\neq p^*_i \text{ for at least one } i$

Given by Pearson's chi-square statistic : $$X^2_{obs} = \sum_{i=1}^k \frac{(n_i - np_i^*)^2}{np^*_i} = \sum_{i=1}^k \frac{(O_i - E_i)^2}{E_i}$$
Where $O_i:=$observed and $E_i:=$expected. Distribution of $\chi^2$ under $\mathcal{H}_0$ is $\chi^2_{(k-1)}$, $(k-1):=$(deg. of freedom). Given $\alpha$, $RR = \{X^2_{obs}> \chi^2_{\alpha, (k-1)}\}$ and $p= Pr\{\chi^2_{(k-1)} > X^2_{obs} \} $. Every expected count must be $\geq 5$ for this test.

\subsubsection{Contingency tables}
$n_{j \bullet} = n_{j1} + ... + n_{jc}$, $n_{\bullet k}=n_{1k}+...+n_{rk}$. 
$n_{1\bullet} +...+ n_{r\bullet} = n_{\bullet 1} + ... + n_{\bullet c}  = n$ (sum of all entries). 

We want to test :

$\mathcal{H}_0$ : $X,Y$ independent \textit{vs} $\mathcal{H}_a$ : $X, Y$ not independent. The expected counts are given by $\hat{E}_{jk} = n \hat{p}_{j\bullet} \hat{p}_{\bullet k} = n_{j\bullet} n_{\bullet k} /n$, where $\hat{p}_{j\bullet} = n_{j\bullet} /n$, $\hat{p}_{\bullet k } = n_{\bullet k} /n$
$$ X^2 = \sum_{j= 1}^r \sum_{k=1}^c \frac{(n_{jk} - (n_{j\bullet} n_{\bullet k} /n ))^2 }{ n_{j\bullet} n_{\bullet k} / n } =\sum_{j= 1}^r \sum_{k=1}^c \frac{(n_{jk} - \hat{E}_{jk})^2 }{\hat{E}_{jk}} $$

Under $\mathcal{H}_0$: $X^2 \sim \chi^2_{(r-1)(c-1)} $, $RR = \{X^2 >  \chi^2_{\alpha, (r-1)(c-1)}\} $

Caveats: must have expected cell count $\geq 5$ for all cells, observations must be mutually independent and identically distributed.

\subsubsection{Fisher's exact test} 
If expected cell count is not $\geq 5$ for all cells.
\subsubsection{McNemar's test}
Matched pairs experiments, e.g. 
\begin{tabular}{c|ccc} 
Response 1/2 & Yes & No & Total \\
\hline Yes & $n_{11}$ & $n_{12}$ & $n_1 \bullet$ \\
No & $n_{21}$ & $n_{22}$ & $n_{2 \bullet}$ \\
\hline Total & $n_{\bullet 1}$ & $n_{\bullet 2}$ & $n$
\end{tabular}

Want to test whether the proportions are the same before and after, i.e. $\mathcal{H}_0: p_1 = p_2$, $\mathcal{H}_a: p_1 \neq p_2$.
$$Q_M = \frac{ (n_{12} -n_{21} )^2}{n_{12} +n_{21} } , \quad \quad RR = \{ Q_M > \chi^2_{\alpha, 1}\} $$

\subsection{Non-Parametric statistics}
\subsubsection{Wilcoxon test}
To test the hypothesis that the probability distributions of both pop. are equivalent $(D_0 = D_1)$.

Conditions: independent samples, cts distributions.

1) order together the observations from both samples

2) assign a rank to each, if equality, take average of ranks.

3) take sum of ranks of each group, let $T$ the sum of sample with smaller size.

$\mathcal{H}_0: D_0 = D_1$, $\mathcal{H}_a: (T_U, T_L $ are table values),

 1) $D_1$ left of $D_2$: $RR=\{T\leq T_L\}$ if $T=T_1$, $\{T\geq T_U\}$ o.w. 
 
 2) $D_1$ right of $D_2$: $RR=\{T \geq T_U\}$ if $T=T_1$, $\{T\leq T_L\}$ o.w. 
 
 3) one or two.

\subsubsection{Normal approx. of Wilcoxon}
If $n_1, n_2 \geq 10$. $Z  = \frac{T_1-(n_1 (n_1 +n_2+1 ) /2)}{\sqrt{n_1 n_2 (n_1 + n_2 +1)/ 12}}$, where $T_1$ sum of ranks corresponding to $D_1$, $n_i$ sample size. Then, $Z\sim \mathcal{N}(0,1)$ and, letting $z_\alpha$ the value from normal table:

(1) $RR = \{Z < -z_\alpha\}; p-value=Pr(Z < Z_{obs} )$

(2) $RR = \{Z > z_\alpha\}; p-value=Pr(Z > Z_{obs} )$

(3) $RR = \{|Z| > z_{\alpha/2}\}; p-value=2 \times Pr(Z > |Z_{obs}|)$

\subsubsection{Wilcoxon's signed rank test (for paired data)}
Let $X_1, ..., X_n$, $Y_1, ... , Y_n$ random samples of paired observations. $Diff_1 = X_1 - Y_1, ..., Diff_n = X_n -Y_n$

1) order absolute values of differences, take out the zeros

2) rank the differences, ties handled as usual

3) let $T_+$, $T_-$ sum of ranks of positive and negative differences

$\mathcal{H}_0: D_1= D_2$, $\mathcal{H}_a:$ same as before, with $T_0$ table value: 

1) $RR = \{ T_+ \leq T_0\}$ 2) $RR = \{ T_- \leq T_0\}$ 

3) $RR = \{ T \leq T_0 \}$ where $T=$ the smallest of $T_-$, $T_+$

\subsubsection{Large sample Wilcoxon's signed rank test}
If the number of pairs $n\geq 25$ (after excluding zeros), then $Z = \frac{T_+ -( n( n+1 ) /4 )}{\sqrt{n(n+1) (2n+1) /24}}$, $Z\sim \mathcal{N}(0,1)$, $RR$ same as 2 sections above.

\subsubsection{Kruskal-Wallis test (CRD)}
Rank-based non-parametric test to test difference in distribution among $\geq 2$ groups.
$\mathcal{H}_0:$ the $k$ distributions are identical, $\mathcal{H}_a:$ at least one differs.

1) Take ranks, as with \textit{Wilcoxon}.

2) Let $\overline{R}_j$ be the rank average of group $j$, $\overline{R}$ overall avg.

Under $\mathcal{H}_0$, we expect $\overline{R}_1 \approx ... \approx \overline{R}_k$. 

$$KW = \frac{12}{ n(n+1) } \sum_{j=1}^k n_j(\overline{R}_j - \overline{R} )^2 \quad \quad RR=\{ KW>\chi^2_{\alpha, k-1} \}$$

\subsubsection{Friedman test (RBD)}
A matched set of $B$ blocks are formed with each consisting of $K$ experimental units. One experimental unit from each block is randomly assigned to each treatment.

1) Rank the observations within blocks.

2) $\overline{R}_j$ average of ranks within treatment j.

Total average of ranks is $K(K+1)/2$, under $\mathcal{H}_0$, we expect $\overline{R}_1 \approx ... \approx \overline{R}_K$. 
$$F_r = \frac{12 B} {K(K+1)} \sum^K_{j=1} (\overline{R}_j - \overline{R})^2 \quad \quad RR = \{ F_r > \chi^2_{\alpha, K-1}\}$$








\end{multicols}
\end{document}








